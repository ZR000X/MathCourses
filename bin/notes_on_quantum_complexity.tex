\documentclass{article}
\usepackage[utf8]{inputenc}
\usepackage{amsthm}
\usepackage{breqn}
\usepackage{enumitem}
\usepackage{tcolorbox}
\usepackage{hyperref}
\title{Notes on Quantum Complexity}
\author{S.G. Schoeman}
\date{August, 2021}
\usepackage{physics}
\usepackage{amsfonts}
\newtheorem{definition}{Definition}
\newtheorem{fact}[definition]{Fact}
\newtheorem{formula}[definition]{Formula}
\newtheorem{equations}[definition]{Equations}
\newtheorem{External Info}[definition]{External Info}
\newtheorem{treatment}[definition]{Treatment}
\begin{document}

\maketitle

\begin{tcolorbox}[title=Definition: Linear Vector Space]\begin{definition}[Linear Vector Space]\label{0}There are these axioms, you see... 
 
 This definition has no references is thus rank $0$.\end{definition}\end{tcolorbox}
\begin{tcolorbox}[title=Definition: Field]\begin{definition}[Field]\label{1}A field is the numbers over which a vector space is defined.
 
 This definition references: (\ref{0}), and is thus rank 1.\end{definition}\end{tcolorbox}
\begin{tcolorbox}[title=Definition: Linear Independence of Vectors]\begin{definition}[Linear Independence of Vectors]\label{2}A set of vectors $\mathbb{V}$ is said to be linearly independent if the only such
linear relation $$a_1v_1+a_2v_2+\ldots+a_nv_n=0,$$ where all $v_i\in\mathbb{V}$ and all
$a_i\in\mathbb{C}$, is the trivial one with all $a_i = 0$. If the set of vectors
is not linearly independent, we say they are linearly dependent. 
 
 This definition references: (\ref{0}), and is thus rank 1.\end{definition}\end{tcolorbox}
\begin{tcolorbox}[title=Definition: Vector Space Dimension]\begin{definition}[Vector Space Dimension]\label{3}The dimension of a vector space is equal to the minimum number of linearly independent vectors it requires to be spanned.
 
 This definition references: (\ref{0}), (\ref{2}), and is thus rank 2.\end{definition}\end{tcolorbox}
\begin{tcolorbox}[title=Fact]\begin{fact}[]\label{4}Any vector $\ket{V}$ in an $n$-dimensional space can be written as a
linear combination of $n$ linearly independent vectors $\ket{1}\ldots\ket{n}$.
 
 This fact references: (\ref{0}), (\ref{2}), (\ref{3}), and is thus rank 3.\end{fact}\end{tcolorbox}
\begin{tcolorbox}[title=Definition: Vector Basis]\begin{definition}[Vector Basis]\label{5}A set of $n$ linearly independent vectors in an $n$-dimensional space
 is called a basis.
 
 This definition references: (\ref{0}), (\ref{2}), (\ref{3}), and is thus rank 3.\end{definition}\end{tcolorbox}
\begin{tcolorbox}[title=Definition: Coordinates of a Vector w.r.t. a Basis]\begin{definition}[Coordinates of a Vector w.r.t. a Basis]\label{6}The coordinates of a vector $\ket{v}$ in a basis $\{\ket{j_i}:i\in I\}$
 are the cooefficients of the expansion $$\ket{v}=v_1\ket{j_1}+\ldots+v_n\ket{j_n}$$ 
 of $\ket{v}$ in that basis.
 
 This definition references: (\ref{5}), (\ref{4}), and is thus rank 4.\end{definition}\end{tcolorbox}
\begin{tcolorbox}[title=Fact]\begin{fact}[]\label{7}The Hamiltonian $H$ of a spring-mass harmonic oscillator is given by\begin{equation}H=\frac{p^2}{2m}+\frac{\omega^2x^2}{2}\end{equation}
 where $\omega^2=k/m$ is the classical frequency of the oscillating system.
 
 This fact has no references is thus rank $0$.\end{fact}\end{tcolorbox}
\begin{tcolorbox}[title=Fact]\begin{fact}[]\label{8}The inverted harmonic oscillator changes the sign of the harmonic oscillator,
 so it has Hamiltonian\begin{equation}H=\frac{p^2}{2m}-\frac{\omega^2x^2}{2}\end{equation}
 where $\omega^2=k/m$ is the classical frequency of the oscillating system.
 
 This fact references: (\ref{7}), and is thus rank 1.\end{fact}\end{tcolorbox}
\begin{tcolorbox}[title=Fact]\begin{fact}[]\label{9}Put $\omega^2=m^2-\lambda$, then $\lambda<m^2$ describes the
 regular harmonic oscillator and $\lambda>m^2$ describes the inverted one, while
 $m^2=\lambda$ is simply a free particle; that is, a particle such that its potential energy
 is independent of its position.
 
 This fact references: (\ref{7}), (\ref{8}), and is thus rank 2.\end{fact}\end{tcolorbox}
\begin{tcolorbox}[title=Formula]\begin{formula}[]\label{10}Consider with the following system. 
 $$\psi(x,t)=\mathcal{N}(t)\exp\left(-\frac{1}{2}\omega_r x^2\right)$$
 with initial conditions $\psi(x,0)=\psi_0,~\mathcal{N}(0)=\mathcal{N}_0$, 
 and where $\omega_r=m$.
 
 This formula has no references is thus rank $0$.\end{formula}\end{tcolorbox}
\begin{tcolorbox}[title=Definition: Hermition Operator]\begin{definition}[Hermition Operator]\label{11}A Hermitian operator is an operator that is equal 
 to its own conjugate transpose.
 
 This definition has no references is thus rank $0$.\end{definition}\end{tcolorbox}
\begin{tcolorbox}[title=Definition: Hermition Matrix]\begin{definition}[Hermition Matrix]\label{12}A Hermitian matrix is a complex square 
 matrix that is equal to its own conjugate transpose.
 
 This definition has no references is thus rank $0$.\end{definition}\end{tcolorbox}
\begin{tcolorbox}[title=Equations]\begin{equations}[]\label{13}The equations of motion for the oscillator are $$\dot{x}=
 \frac{\partial H}{\partial p}=\frac{p}{m},~~~~\dot{p}=-\frac{\partial H}{\partial x}=
 -m\omega^2 x$$ which, through elimination of $\dot{p}$, becomes
 $$\ddot{x}=-\omega^2x$$ which easily solves by
 $$x(t)=A\cos(\omega t+\phi)$$ where $x_0=x(0)=A\cos(\phi)$. Note then that
 $$E=T+V=\frac{1}{2}m\dot{x}^2+\frac{1}{2}m\omega^2x^2=\frac{1}{2}mA^2\omega^2$$
 where $E$ is the energy of the classical system.
 
 This equations references: (\ref{7}), and is thus rank 1.\end{equations}\end{tcolorbox}
\begin{tcolorbox}[title=Fact]\begin{fact}[]\label{14}The equation for a quantum oscillator with state $\ket{\psi}$ is
 $$i\hbar\frac{d}{dt}\ket{\psi}=H\ket{\psi}$$
 
 This fact has no references is thus rank $0$.\end{fact}\end{tcolorbox}
\begin{tcolorbox}[title=External Info: AdS/CFT correspondence]\begin{External Info}[AdS/CFT correspondence]\label{15}
 
 This External Info references: \{\href{https://en.wikipedia.org/wiki/AdS/CFT_correspondence}{link}\}, [\ref{3}], and is thus rank $\infty$.\end{External Info}\end{tcolorbox}
\begin{tcolorbox}[title=External Info: Einstein-Rosen Bridge (Wormhole)]\begin{External Info}[Einstein-Rosen Bridge (Wormhole)]\label{16}
 
 This External Info references: \{\href{https://en.wikipedia.org/wiki/Wormhole}{link}\}, [\ref{3}], and is thus rank $\infty$.\end{External Info}\end{tcolorbox}
\begin{tcolorbox}[title=External Info: Complexity Equals Action]\begin{External Info}[Complexity Equals Action]\label{17}
 
 This External Info references: [\ref{5}], and is thus rank $\infty$.\end{External Info}\end{tcolorbox}
\begin{tcolorbox}[title=Definition: Vector Space]\begin{definition}[Vector Space]\label{18}A vector space (also called a linear space) is a set of objects called vectors, which may be added together and multiplied ("scaled") by numbers, called scalars. They obey specific axioms...
 
 This definition references: (\ref{0}), \{\href{https://en.wikipedia.org/wiki/Vector_space}{link}\}, and is thus rank $\infty$.\end{definition}\end{tcolorbox}
\begin{tcolorbox}[title=Definition: Binary Operation]\begin{definition}[Binary Operation]\label{19}In mathematics, a binary operation or dyadic operation is a calculation that combines two elements (called operands) to produce another element. More formally, a binary operation is an operation of arity two.
 
 This definition references: \{\href{https://en.wikipedia.org/wiki/Binary_operation}{link}\}, and is thus rank $\infty$.\end{definition}\end{tcolorbox}
\begin{tcolorbox}[title=Definition: Inner Product Space]\begin{definition}[Inner Product Space]\label{20}An inner product space, or a Hausdorff pre-Hilbert space, is a vector space with a binary operation called an inner product.
 
 This definition references: (\ref{18}), (\ref{19}), \{\href{https://en.wikipedia.org/wiki/Inner_product_space}{link}\}, and is thus rank $\infty$.\end{definition}\end{tcolorbox}
\begin{tcolorbox}[title=Definition: Complete Metric Space]\begin{definition}[Complete Metric Space]\label{21}In mathematical analysis, a metric space M is called complete (or a Cauchy space) if every Cauchy sequence of points in M has a limit that is also in M.
 
 This definition references: \{\href{https://en.wikipedia.org/wiki/Complete_metric_space}{link}\}, and is thus rank $\infty$.\end{definition}\end{tcolorbox}
\begin{tcolorbox}[title=Definition: Hilbert Space]\begin{definition}[Hilbert Space]\label{22}A Hilbert space H is a real or complex inner product space that is also a complete metric space with respect to the distance function induced by the inner product.
 
 This definition references: (\ref{20}), (\ref{21}), \{\href{https://en.wikipedia.org/wiki/Hilbert_space#Definition_and_illustration}{link}\}, and is thus rank $\infty$.\end{definition}\end{tcolorbox}
\begin{tcolorbox}[title=Definition: Unitary Operator]\begin{definition}[Unitary Operator]\label{23}In functional analysis, a unitary operator is a surjective bounded operator on a Hilbert space preserving the inner product. Unitary operators are usually taken as operating on a Hilbert space, but the same notion serves to define the concept of isomorphism between Hilbert spaces.
 
 This definition references: \{\href{https://en.wikipedia.org/wiki/Unitary_operator}{link}\}, (\ref{22}), (\ref{20}), and is thus rank $\infty$.\end{definition}\end{tcolorbox}
\begin{tcolorbox}[title=Treatment: Quantum Treatment of an Inverted Harmonic Oscillator (IHO)]\begin{treatment}[Quantum Treatment of an Inverted Harmonic Oscillator (IHO)]\label{24}A reference state $\ket{\psi_R}$ is transformed, via a unitary operator $\hat{\mathcal{U}}$, into target state $\ket{\psi_T}$: $$\ket{\psi_T}=\hat{\mathcal{U}}\ket{\psi_R}$ Now, say that the operator is but an evolving operator $\hat{\mathcal{U}}=e^{-iHt}$, so $$\ket{\psi_T}=e^{-iHt}\ket{\psi_R}$$
 
 This treatment references: (\ref{23}), and is thus rank $\infty$.\end{treatment}\end{tcolorbox}
\section*{External References}
\begin{enumerate}[label={[\arabic*]}]\item A. Bhattacharyya, S. Das, S.S. Haque, B. Underwood. \begin{textit}Cosmological Complexity. \end{textit}(2020). arXiv:2001.08664
\item R. Jefferson, R.C. Myers. \begin{textit}Circuit complexity in quantum field theory. \end{textit}(2017). arXiv:1707.08570
\item T. Ali, A. Bhattacharyya, S.S. Haque, E.H. Kim, N. Moynihan, J. Murugan. \begin{textit}Chaos and Complexity in Quantum Mechanics. \end{textit}(2020). arXiv:1905.13534
\item T. Ali, A. Bhattacharyya, S.S. Haque, E.H. Kim, N. Moynihan. \begin{textit}Time Evolution of Complexity: A Critique of Three Methods. \end{textit}(2019). arXiv:1810.02734
\item S.S. Haque, C. Jana, B. Underwood. \begin{textit}Saturation of Thermal Complexity of Purification. \end{textit}(2021). arXiv:2107.08969
\item A.R. Brown, D.A. Roberts, L. Susskind, B. Swingle, Y. Zhao. \begin{textit}Complexity Equals Action. \end{textit}(2016). arXiv:1509.07876
\end{enumerate}
\end{document}
